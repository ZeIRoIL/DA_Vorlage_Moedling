
  %% diplomarbeit.tex
  %% Copyright 2015 Simon M. Laube
  %
  % This work may be distributed and/or modified under the
  % conditions of the LaTeX Project Public License, either version 1.3
  % of this license or (at your option) any later version.
  % The latest version of this license is in
  %   http://www.latex-project.org/lppl.txt
  % and version 1.3 or later is part of all distributions of LaTeX
  % version 2005/12/01 or later.
  %
  % This work has the LPPL maintenance status `author maintained'.
  % 
  % The Current Maintainer of this work is S. M. Laube
  %
  % This work consists of the files listed in ./Help/files.txt

%%=====================================================%%
%% Neues Diplomarbeitstemplate der ET	   			   %%
%% Abteilung ab 2013/2014				   			   %%
%% Erstellt von Simon Michael Laube		   			   %%
%% Betreut von  Prof. Mag. Dipl.-Ing. Dr. Daniel Asch  %%
%%			    Prof. Dipl.-Ing. Dr. Wilhelm Haager	   %%
%%=====================================================%%
%% Dokumentklasse KOMA-Script Report
\documentclass[paper=a4,12pt]{scrreprt}
% Encoding UTF8
\usepackage[utf8]{inputenc}
% 8 Bit Aufloesung der Buchstaben
\usepackage[T1]{fontenc}
% Seitenraender
\usepackage[scale=0.72]{geometry}
% Spracheinstellungen
\usepackage[english, naustrian]{babel} % your native language must be the last one!!
% erweiterte Farbenpalette
\usepackage[dvipsnames]{xcolor}
% Abbildungen
\usepackage{graphicx}
% Tabellen (erweitert)
\usepackage{tabularx}
% TikZ + Circuit-TikZ (fuer Schaltungen)									
\usepackage[europeanresistors,							
			europeaninductors]{circuitikz}
% Nuetzliche TikZ Libraries
\usetikzlibrary{arrows,automata,positioning}
% Mathematikpakete!
\usepackage{amsmath,amssymb}							
%\usepackage{mathtools}	
% PDF Einbindung (zB Datenblaetter)
\usepackage{pdfpages}
% Source Code Einbindung, Setup siehe:
% http://en.wikibooks.org/wiki/LaTeX/Source_Code_Listings									
\usepackage{listings,scrhack} %scrhack vermeidet Umschaltung auf KOMA Floats..			
% Ausrichtung der Bilder
\usepackage[export]{adjustbox}

% The Euro­pean cur­rency sym­bol for the Euro			
\usepackage{eurosym}

% Landscape
\usepackage{lscape}
\usepackage{tabularx}
% Diplomarbeitsspezifisches Package etdipa
\usepackage{etdipa}

%% Abkuerzungsverzeichnis
\usepackage[]{acronym}
\usepackage{xcolor}
%% Todos
\usepackage[]{todonotes}

%% Ganttdiagramme
\usepackage{pgfgantt}
\usepackage{xcolor}
\usepackage{graphics}
\usepackage{graphicx}
\usepackage{capt-of}
%% Subfigures
\usepackage[lofdepth]{subfig}


%%==== Definitionen fuer die Diplomarbeit ============%%
\dokumenttyp{DIPLOMARBEIT}
\title{Musterdokument}
\author{Hans Huber \and Birgit Bauer}
\date{\today}
\place{Mödling}
\schuljahr{2018/19}
\professor{Prof. Rick}
\dipacolor{ETred}
%%====================================================%%


% Hyperlinks im Dokument
\usepackage[colorlinks=true,
			linkcolor=black,
			citecolor=black,
			bookmarks=true,
			urlcolor=black,
			bookmarksopen=true]{hyperref}
\begin{document}

\frontmatter

%%================ Titelseite ==========================%%
\maketitle
% Verantwortliche/Verfasser
%%======================================================%%

%%================ Eidesstattliche Erklaerung ==========%%

\begin{Eid}%Unterschrift der Diplomanden hinzufuegen!
\unterschrift{Hans Huber}
\unterschrift{Birgit Bauer}

\end{Eid}\newpage
%%======================================================%%

%%================ Diplomandenvorstellung ==============%%
\input{Textparts/diplomanden.tex}
%%======================================================%%

%%================ Tabelle ==============%%
\input{Textparts/tabellen.tex}
%%======================================================%%

%%================ Danksagungen ========================%%
\newpage
\input{Textparts/danksagungen.tex}
%%======================================================%%

%%================ Abstract /Zusammenfass. =============%%
\input{Textparts/abstract.tex}
\input{Textparts/zusammenfassung.tex}
%\selectlanguage{english} % necessary for English speaking users
% delete this line if your native language is German 
%%======================================================%%

%%================ Inhaltsverzeichnis ==================%%
\tableofcontents
%%======================================================%%

% Verantwortliche/Verfasser
\responsible{Birgit Bauer}

%Ab hier Hauptteil
\mainmatter


%%======================================================%%
\input{Textparts/Kapitel3.tex}
\input{Textparts/Kapitel4.tex}
\input{Textparts/Kapitel6.tex}
\input{Textparts/Kapitel7.tex}
\input{Textparts/Kapitel8.tex}
\input{Textparts/Kapitel9.tex}
\input{Textparts/Kapitel10.tex}
\input{Textparts/Kapitel11.tex}
\input{Textparts/Kapitel14.tex}
%\input{Textparts/Kapitel15.tex}

\appendix

%%================ Abkuerzungsverzeichnis ==============%%
\input{Textparts/abkuerzungen.tex}
%%======================================================%%


%%================ Abbildungsverzeichnis ===============%%
\setcounter{lofdepth}{2}
\dipalistoffigures
%%======================================================%%

%%================ Tabellenverzeichnis  ================%%
\setcounter{lotdepth}{2}
\dipalistoftables
%%======================================================%%

%%================ Literaturverzeichnis ================%%
\newpage
\input{Textparts/literatur.tex}
 
	 
%%======================================================%%
\end{document}
