
\chapter{Begleitprotokoll}
In einem Begleitprotokoll sind der Arbeitsablauf $($zeitliche Auflistung, wann und wie lange an der abschließenden Arbeit gearbeitet wurde$)$ sowie die verwendeten Hilfsmittel und Hilfestellungen zu dokumentieren. Jedes Teammitglied ist verpflichtet, selbstständig sein eigenes Begleitprotokoll zu führen. Das Begleitprotokoll ist der schriftlichen Arbeit beizulegen $($§ 9 Abs. 2 Prüfungsordnung BMHS$)$.
\\
\noindent In der Rubrik Erstellung finden Sie eine Begleitprotokoll-Vorlage sowie Erläuterungen zum Begleitprotokoll. Sprechen Sie aber mit Ihrem Betreuer/Ihrer Betreuerin, ob Sie dieses Begleitprotokoll als Vorlage verwenden können.\\
\begin{large} Quelle:\url{http://www.diplomarbeiten-bbs.at/faq/faq-schuelerinnen}
\end{large}
\\
\noindent
Im Begleitprotokoll, das als Nachweis von Tätigkeiten, Meetings und Entscheidungen während der Diplomarbeit gilt, sind laufend Aufzeichnungen von den Schülerinnen bzw. von den Schülern zu führen. 
Dazu gibt es mehrere Möglichkeiten: 
\begin{itemize}
	\item das auf der DA-Webseite $($http://www.dipolmarbeiten-bbs.at$)$ vorgeschlagene Formular „Begleitprotokoll“ oder 
	\item die Projektmanagement Tools $($mit Taskverwaltung, Zeittracking und Meeting-Protokollen$)$ oder 
	\item die digitale Ablage in einem Dokumentenverwaltungssystem $($z. B. Dropbox usw.$)$ 
\end{itemize}
Die gewählte Form ist mit der Betreuerin bzw. dem Betreuer abzuklären und beinhalte folgende Aufzeichnungen: 
\begin{itemize}
	\item Dokumentation wichtiger Entscheidungen und Ereignisse, 
	\item Teambesprechungen deren Inhalte und Beschlüsse, 
	\item Besprechungen mit Betreuerinnen und Betreuern, 
	\item Dokumentation des individuellen Zeitaufwandes, 
	\item Kontakt zu Sponsoren, Investoren und Partnern. 	
\end{itemize}
Alle Inhalte müssen korrekt und vollständig dokumentiert sein. Auf Wunsch der Betreuerin bzw. des Betreuers sind die Aufzeichnungen jederzeit vorzulegen. 
Diese Aufzeichnungen dienen als: 
\begin{itemize}
	\item Nachweis von Tätigkeiten und Besprechungen, 
	\item Nachweis der Betreuungstätigkeit, 
	\item Überblick und Nachvollziehbarkeit von wichtigen Entscheidungen, 
	\item Nachvollziehbarkeit des Informationsflusses.
\end{itemize}
Quelle:	\url{http://www.diplomarbeiten-bbs.at/erstellung}\\
{\large{\textbf{Vorschlag $\rightarrow$}} Monatliche Zeit-Übersicht auf Basis der Wochenberichte
\section{Begleitprotokoll $<$Schüler 1$>$}
\begin{tabular}
{|c|l|r|}
\hline
{\textbf{Zeitraum}} &  Arbeiten / Tätigkeiten / Meetings / … & Stunden\\
\hline
2019/08 & & \\
\hline
2019/09 & & \\
\hline
2019/10 & & \\
\hline
2019/11 & & \\
\hline
2019/12 & & \\
\hline
2020/01 & & \\
\hline
2020/02 & & \\
\hline
2020/03 & & \\
\hline
2020/04 & & \\
\hline
\end{tabular}
\section{Begleitprotokoll $<$Schüler 2$>$}
\section{Begleitprotokoll $<$Schüler 3$>$}
